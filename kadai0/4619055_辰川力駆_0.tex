\documentclass[12pt]{jarticle}
\usepackage{TUSIreport}
\usepackage{otf}
\usepackage{multirow}
\usepackage[normalem]{ulem}
\useunder{\uline}{\ul}{}
%%%%%%%%%%%%%%%%%%
\begin{document}
%%%%%%%%%%%%%%%%%%%%%%%%%%%%%%%%%%%%%%%%%%%%%%%%%%%%%%%%
% 表紙を出力する場合は,\提出者と\共同実験者をいれる
% \提出者{科目名}{課題名}{提出年}{提出月}{提出日}{学籍番号}{氏名}
% \共同実験者{一人目}{二人目}{..}{..}{..}{..}{..}{八人目}
%%%%%%%%%%%%%%%%%%%%%%%%%%%%%%%%%%%%%%%%%%%%%%%%%%%%%%%
\提出者{情報工学実験3}{課題0}
{2021}{4}{15}{4619055}{辰川力駆}
%%%%%%%%%%%%%%%%%%%%%%%%%%%%%%%%%%%%%%%%%%%%%%%%%%%%%%%%%
\共同実験者{}{}{}{}{}{}{}{}
%%%%%%%%%%%%%%%%%%%%%%%%%%%%%%%%%%%%%%%%%%%%%%%%%%%%%%%%%
% 表紙を出力する場合はコメントアウトしない
%%%%%%%%%%%%%%%%%%%%%%%%%%%%%%%%%%%%%%%%%%%%%%%%%%%%%%%%%
%\表紙出力
%%%%%%%%%%%%%%%%%%%%%%%%%%%%%%%%%%%%%%%%%%%%%%%%%%%%%%%
% 以下はレポート本体,reportmain.tex に書いてある.
% \inputを使っているが,直接書いても良い.
%%%%%%%%%%%%%%%%%%%%%%%%%%%%%%%%%%%%%%%%%%%%%%%%%%%%%%%

\begin{center}
    学籍番号: \underline{4619055},氏名: \underline{辰川 力駆}
\end{center}

以下の表は, 東京理科大学工学部情報工学科の2年生向けに開講されている
情報工学実験1のスケジュールを, \LaTeX により作成したものである.
作成のためには, 以下のスタイルファイルをusepackage するとよい.

\begin{itemize}
    \item array.sty
    \item otf.sty(ただし, deluxe オプションをつける)
    \item multirow.sty
    \item multicolumn.sty
\end{itemize}

\begin{table}[htb]
    \begin{center}
        \begin{tabular}{|l||c|c|c|c|c|c||c|c|c|c|c|c|}
            \hline
            実験日      & G1                                                           & G2                               & G3                               & G4                               & G5                               & G6                               & G7                               & G8                               & G9                               & G10                              & G11                              & G12                              \\ \hline \hline
            4月12日(月) & \multicolumn{12}{c|}{ガイダンス, \LaTeX 演習(\underline{0})}                                                                                                                                                                                                                                                                                                                                                                                                  \\ \hline \hline
            4月19日(月) & \multirow{3}{*}{{\underline{2}}}                             & \multirow{3}{*}{{\underline{2}}} & \multirow{3}{*}{{\underline{2}}} & \multirow{3}{*}{{\underline{3}}} & \multirow{3}{*}{{\underline{3}}} & \multirow{3}{*}{{\underline{3}}} & \multirow{2}{*}{{\underline{1}}} & \multirow{2}{*}{{\underline{1}}} & \multirow{2}{*}{{\underline{5}}} & \multirow{2}{*}{{\underline{5}}} & \multirow{2}{*}{{\underline{4}}} & \multirow{2}{*}{{\underline{4}}} \\
            4月26日(月) &                                                              &                                  &                                  &                                  &                                  &                                  &                                  &                                  &                                  &                                  &                                  &                                  \\ \cline{8-13}
            5月10日(月) &                                                              &                                  &                                  &                                  &                                  &                                  & \multirow{2}{*}{{\underline{4}}} & \multirow{2}{*}{{\underline{4}}} & \multirow{2}{*}{{\underline{1}}} & \multirow{2}{*}{{\underline{1}}} & \multirow{2}{*}{{\underline{5}}} & \multirow{2}{*}{{\underline{5}}} \\ \cline{2-7}
            5月17日(月) & \multirow{3}{*}{{\underline{3}}}                             & \multirow{3}{*}{{\underline{3}}} & \multirow{3}{*}{{\underline{3}}} & \multirow{3}{*}{{\underline{2}}} & \multirow{3}{*}{{\underline{2}}} & \multirow{3}{*}{{\underline{2}}} &                                  &                                  &                                  &                                  &                                  &                                  \\ \cline{8-13}
            5月24日(月) &                                                              &                                  &                                  &                                  &                                  &                                  & \multirow{2}{*}{{\underline{5}}} & \multirow{2}{*}{{\underline{5}}} & \multirow{2}{*}{{\underline{4}}} & \multirow{2}{*}{{\underline{4}}} & \multirow{2}{*}{{\underline{1}}} & \multirow{2}{*}{{\underline{1}}} \\
            5月31日(月) &                                                              &                                  &                                  &                                  &                                  &                                  &                                  &                                  &                                  &                                  &                                  &                                  \\ \hline \hline
            6月07日(月) & \multirow{2}{*}{{\underline{1}}}                             & \multirow{2}{*}{{\underline{1}}} & \multirow{2}{*}{{\underline{5}}} & \multirow{2}{*}{{\underline{5}}} & \multirow{2}{*}{{\underline{4}}} & \multirow{2}{*}{{\underline{4}}} & \multirow{3}{*}{{\underline{2}}} & \multirow{3}{*}{{\underline{2}}} & \multirow{3}{*}{{\underline{2}}} & \multirow{3}{*}{{\underline{3}}} & \multirow{3}{*}{{\underline{3}}} & \multirow{3}{*}{{\underline{3}}} \\
            6月14日(月) &                                                              &                                  &                                  &                                  &                                  &                                  &                                  &                                  &                                  &                                  &                                  &                                  \\ \cline{2-7}
            6月21日(月) & \multirow{2}{*}{{\underline{4}}}                             & \multirow{2}{*}{{\underline{4}}} & \multirow{2}{*}{{\underline{1}}} & \multirow{2}{*}{{\underline{1}}} & \multirow{2}{*}{{\underline{5}}} & \multirow{2}{*}{{\underline{5}}} &                                  &                                  &                                  &                                  &                                  &                                  \\ \cline{8-13}
            6月28日(月) &                                                              &                                  &                                  &                                  &                                  &                                  & \multirow{3}{*}{{\underline{3}}} & \multirow{3}{*}{{\underline{3}}} & \multirow{3}{*}{{\underline{3}}} & \multirow{3}{*}{{\underline{2}}} & \multirow{3}{*}{{\underline{2}}} & \multirow{3}{*}{{\underline{2}}} \\ \cline{2-7}
            7月05日(月) & \multirow{2}{*}{{\underline{5}}}                             & \multirow{2}{*}{{\underline{5}}} & \multirow{2}{*}{{\underline{4}}} & \multirow{2}{*}{{\underline{4}}} & \multirow{2}{*}{{\underline{1}}} & \multirow{2}{*}{{\underline{1}}} &                                  &                                  &                                  &                                  &                                  &                                  \\
            7月12日(月) &                                                              &                                  &                                  &                                  &                                  &                                  &                                  &                                  &                                  &                                  &                                  &                                  \\ \hline \hline
            7月19日(月) & \multicolumn{12}{c|}{予備日}                                                                                                                                                                                                                                                                                                                                                                                                                                  \\ \hline
        \end{tabular}
    \end{center}
\end{table}

\begin{flushright}
    (注)表中の下線付数字は課題番号である.
\end{flushright}



%%%%%%%%%%%%%%%%%%%%%%%%%%%%%%%%%%%%%%%%%%%%%%%%%%%%%%%
\end{document}
