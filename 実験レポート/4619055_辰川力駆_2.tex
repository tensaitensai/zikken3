\documentclass[12pt]{jarticle}
\usepackage{TUSIreport}
\usepackage{otf}
\usepackage[dvipdfmx]{graphicx}
\usepackage[dvipdfmx]{color}
\usepackage{amsmath}
\usepackage{amssymb}
\usepackage{color}
\usepackage{hhline}
\usepackage{fancybox,ascmac}
\usepackage{multirow}
\usepackage{url}
\usepackage{bm}
\usepackage{listings,jlisting}
%%%%%%%%%%%%%%%%%%
\lstdefinestyle{py}{
    language={Python},
    backgroundcolor={\color[gray]{.85}},
    basicstyle={\small},
    identifierstyle={\small},
    commentstyle={\small\ttfamily \color[rgb]{0,0.5,0}},
    keywordstyle={\small\bfseries \color[rgb]{1,0,0}},
    ndkeywordstyle={\small},
    stringstyle={\small\ttfamily \color[rgb]{0,0,1}},
    frame={tb},
    breaklines=true,
    columns=[l]{fullflexible},
    xrightmargin=0zw,
    xleftmargin=3zw,
    numberstyle={\scriptsize},
    stepnumber=1,
    numbersep=1zw,
    morecomment=[l]{//}
}
\begin{document}
%%%%%%%%%%%%%%%%%%%%%%%%%%%%%%%%%%%%%%%%%%%%%%%%%%%%%%%%
% 表紙を出力する場合は,\提出者と\共同実験者をいれる
% \提出者{科目名}{課題名}{提出年}{提出月}{提出日}{学籍番号}{氏名}
% \共同実験者{一人目}{二人目}{..}{..}{..}{..}{..}{八人目}
%%%%%%%%%%%%%%%%%%%%%%%%%%%%%%%%%%%%%%%%%%%%%%%%%%%%%%%
\提出者{情報工学実験3}{課題2 パターン認識}
{2021}{6}{17}{4619055}{辰川力駆}
%%%%%%%%%%%%%%%%%%%%%%%%%%%%%%%%%%%%%%%%%%%%%%%%%%%%%%%%%
\共同実験者{}{}{}{}{}{}{}{}
%%%%%%%%%%%%%%%%%%%%%%%%%%%%%%%%%%%%%%%%%%%%%%%%%%%%%%%%%
% 表紙を出力する場合はコメントアウトしない
%%%%%%%%%%%%%%%%%%%%%%%%%%%%%%%%%%%%%%%%%%%%%%%%%%%%%%%%%
\表紙出力
%%%%%%%%%%%%%%%%%%%%%%%%%%%%%%%%%%%%%%%%%%%%%%%%%%%%%%%
% 以下はレポート本体,reportmain.tex に書いてある.
% \inputを使っているが,直接書いても良い.
%%%%%%%%%%%%%%%%%%%%%%%%%%%%%%%%%%%%%%%%%%%%%%%%%%%%%%%
\section{はじめに}
パターン認識とは、
入力データをあらかじめ定めていた複数のクラスの1つに対応させる処理のことをいう。
分類を行うクラスの種類に応じて、パターン認識は大きく、
「分類問題」と「予測問題」に分けることができる。
また、分類問題、予測問題の両方に関して、
「教師あり学習」と「教師なし学習」の2つが存在しており、
本実験では、入力データと出力の組みが与えられているデータを利用したパターン認識である
「教師あり学習」について理解する。
また、「教師あり学習」の代表的な手法であるロジスティック回帰
とサポートベクトルマシンについて実装する。

\section{レポート課題}
\subsection{課題1}
\subsubsection*{課題内容}
講義のロジスティック回帰モデルでは、全ての変数を用いたが、
モデルに組み込む変数を選択し最適なモデルを選択することができる。
変数選択手法に関してどれか一つ調べる。
\subsubsection*{手順}
\begin{enumerate}
    \item webで変数選択手法について検索する。
    \item その変数選択手法に関してPythonで実装する。
    \item このプログラムを実行する。
\end{enumerate}
\subsubsection*{結果}

変数選択手法として、Wrapper Methodについて調べた。
機械学習モデルを使用して特徴量の組み合わせを評価することで、
変数間の関係を探し出し、
それぞれのモデルに最適な特徴量の組み合わせを探し出す手法である。

それについて実装したのがソースコード1である。結果を比較すると以下の表1のようになる。

\begin{table}[htb]
    \caption{ロジスティクス回帰モデルの変数選択をした場合としていない場合}
    \begin{center}
        \begin{tabular}{|c||c|c|}
            \hline
                      & 変数選択なし(すべて使う) & 変数選択あり \\ \hline \hline
            accuracy  & 0.78125                  & 0.78125      \\ \hline
            汎化誤差  & 0.21875                  & 0.21875      \\ \hline
            precision & 0.72727                  & 0.73810      \\ \hline
            f1 score  & 0.64646                  & 0.63918      \\ \hline
        \end{tabular}
    \end{center}
\end{table}

\clearpage

\subsubsection*{考察}
表1より、変数選択なしより変数選択ありのほうが適合率が少しだけ高かった。
逆にf1 scoreは僅かに変数選択なしのほうが高かった。
変数選択を実装した場合としない場合でほとんど変わらなかったことから、
変数選択を実装しなくても全ての変数を用いた場合で十分に
ロジスティック回帰で良いモデルを作成できると考える。

\subsection{課題2}
\subsubsection*{課題内容}

本実験では線形カーネル関数を用いたサポートベクトルマシンの手法を適用した。
課題2ではそれ以外のカーネル関数を用いたサポートベクトルマシンについて調査し、
実装する。

\subsubsection*{手順}
\begin{enumerate}
    \item webでカーネル関数を用いたサポートベクトルマシンについて検索する。
    \item そのサポートベクトルマシンに関してPythonで実装する。
    \item このプログラムを実行する。
\end{enumerate}

\subsubsection*{結果}
線形カーネル関数以外のサポートベクトルマシンとして、
RBFカーネルについて調べた。
コスト関数$C$とカーネル関数の$\gamma$を用いる方法であり、
$\gamma$の値が小さいほど単純な決定境界となり、
大きいほど複雑な決定境界となる。

それについて実装したのがソースコード2である。
結果を比較すると以下の表2のようになる。

\begin{eqnarray}
    \gamma &=& 0.01 \nonumber\\
    C &=& 1.02 \nonumber
\end{eqnarray}

\begin{table}[htb]
    \caption{線形カーネルとRBFカーネルの結果}
    \begin{center}
        \begin{tabular}{|c||c|c|}
            \hline
                      & 線形カーネル & RBFカーネル \\ \hline \hline
            accuracy  & 0.78125      & 0.74375     \\ \hline
            汎化誤差  & 0.21875      & 0.25625     \\ \hline
            precision & 0.75         & 0.70588     \\ \hline
            f1 score  & 0.63158      & 0.53933     \\ \hline
        \end{tabular}
    \end{center}
\end{table}

\clearpage

\subsubsection*{考察}

RBFカーネルは、
\begin{eqnarray}
    k(x,x') = \text{exp}(-\gamma || x - x' ||^2) \nonumber
\end{eqnarray}
という式で定義される。

表2より、accuracy、precision、f1 scoreのどれにおいても
線形カーネル関数を用いたサポートベクトルマシンの方が性能が高いと分かる。
なので、今回のデータに関していえば線形カーネル関数を用いた
サポートベクトルマシンを実装するほうが良いと考える。

\subsection{課題3}
sklearn.datasetの中にあるワインの品質データ(Wine recognition dataset)を用いて、
今回の実験の手法(ロジスティック、SVM)の中から2値分類を行う。
\subsubsection*{課題内容}
\subsubsection*{手順}
\subsubsection*{結果}
\subsubsection*{考察}
\subsection{課題4}
\subsubsection*{課題内容}
\subsubsection*{手順}
\subsubsection*{結果}
\subsubsection*{考察}

\section{感想}
先週の感想でも述べたが、自分的には動画を1つにまとめてくれる方が見やすかった。
良かった点としては、前回よりは端的にまとめていたので見返すことがしやすかった。
あまり良くなかった点としては、学校に行くのに動画を見るだけなのはあまり良くないと考える。

\begin{thebibliography}{99}
    \label{sannkoubunnkenn_chapter}
    \bibitem[1]{a}
    特徴量選択のまとめ - Qiita

    \url{https://qiita.com/shimopino/items/5fee7504c7acf044a521}

    最終閲覧日2021/06/20

    \bibitem[2]{b}
    変数選択(Feature Selection)の実装と改善の確認 - 学習する天然ニューラルネット

    \url{https://aotamasaki.hatenablog.com/entry/2018/04/27/222536}

    最終閲覧日2021/06/20

    \bibitem[3]{c}
    【SVM】rbfカーネルのハイパーパラメータをグリッドサーチとベイズ最適化で探す【irisデータセット】

    \url{https://kenyu-life.com/2019/08/24/rbf_svm_gs/}

    最終閲覧日2021/06/20
\end{thebibliography}

\clearpage
\appendix
\section{付録}
\begin{lstlisting}[style = py,caption=レポート課題1]
    import pandas as pd
    import numpy as np
    import seaborn as sns
    import matplotlib.pyplot as plt
    from sklearn.model_selection import train_test_split
    from sklearn.metrics import confusion_matrix, accuracy_score, precision_score, recall_score, f1_score
    from sklearn.metrics import roc_curve, auc
    from sklearn.linear_model import LogisticRegression
    from sklearn.feature_selection import RFECV
    
    pima = pd.read_csv("data/pima.csv")
    
    X = pima.iloc[:,0:7]
    Y =pima["type"].map({'No': 0, 'Yes': 1})
    
    select = RFECV(LogisticRegression(), cv=10, scoring='average_precision')
    select.fit(X, Y)
    mask = select.support_
    X_selected = X.iloc[:,mask]
    
    X_tr, X_te, Y_tr, Y_te = train_test_split(X_selected, Y, test_size=0.3, random_state=4619055)
    
    lr = LogisticRegression(C=np.inf,solver='newton-cg') 
    lr.fit(X_tr, Y_tr)
    
    Y_pred = lr.predict(X_te)
    
    print('confusion matrix = \n', confusion_matrix(y_true=Y_te, y_pred=Y_pred))
    print('accuracy = ', accuracy_score(y_true=Y_te, y_pred=Y_pred))
    print('汎化誤差 = ', 1-accuracy_score(y_true=Y_te, y_pred=Y_pred))
    print('precision = ', precision_score(y_true=Y_te, y_pred=Y_pred))
    print('f1 score = ', f1_score(y_true=Y_te, y_pred=Y_pred))
    
    Y_score = lr.predict_proba(X_te)[:, 1]
    fpr, tpr, thresholds = roc_curve(y_true=Y_te, y_score=Y_score)
    
    plt.plot(fpr, tpr, label='roc curve (area = %0.3f)' % auc(fpr, tpr))
    plt.plot([0, 1], [0, 1], linestyle='--', label='random')
    plt.plot([0, 0, 1], [0, 1, 1], linestyle='--', label='ideal')
    plt.legend()
    plt.xlabel('false positive rate')
    plt.ylabel('true positive rate')
\end{lstlisting}


\begin{lstlisting}[style = py,caption=レポート課題2]
    import pandas as pd
    import numpy as np
    from sklearn.model_selection import GridSearchCV
    from sklearn.metrics import classification_report
    from sklearn.svm import SVC
    from sklearn.metrics import accuracy_score
    from sklearn.model_selection import train_test_split
    from sklearn.metrics import confusion_matrix, accuracy_score, precision_score, recall_score, f1_score
    
    pima = pd.read_csv("data/pima.csv")
    X = pima.iloc[:,0:7]
    Y =pima["type"].map({'No': 0, 'Yes': 1})
    X_tr, X_te, Y_tr, Y_te = train_test_split(X, Y, test_size=0.3, random_state=4619055)
    
    model = SVC(C=1 ,kernel='linear', random_state=4619055)
    model.fit(X_tr, Y_tr)
    Y_pred = model.predict(X_te)    
    
    print('confusion matrix = \n', confusion_matrix(y_true=Y_te, y_pred=Y_pred))
    print('accuracy = ', accuracy_score(y_true=Y_te, y_pred=Y_pred))
    print('汎化誤差 = ', 1-accuracy_score(y_true=Y_te, y_pred=Y_pred))
    print('precision = ', precision_score(y_true=Y_te, y_pred=Y_pred))
    print('f1 score = ', f1_score(y_true=Y_te, y_pred=Y_pred))
    
    best_score = float(0.0)
    best_param_gamma = 0.0
    best_param_C = 0.0
     
    scores = pd.DataFrame()
    for gamma in np.linspace(0.01, 10, 100):
        for C in np.linspace(0.01, 10, 100):
            svm = SVC(kernel = 'rbf', gamma=gamma, C=C, random_state=4619055)
            svm.fit(X_tr, Y_tr)
            scores = scores.append(
                    {
                        'gamma': gamma,
                        'C': C,
                        'accuracy': svm.score(X_te, Y_te)
                    },
                    ignore_index=True)
            
            if best_score < svm.score(X_te, Y_te):
                best_score = svm.score(X_te, Y_te)
                best_param_gamma = gamma
                best_param_C = C
                
    print("ベストスコア:", round(best_score,2))
    print('g:%s c:%s' %(round(best_param_gamma,2),round(best_param_C,2)))
    
    gamma = best_param_gamma
    C = best_param_C
     
    svm = SVC(kernel = 'rbf', gamma=gamma, C=C, random_state=4619055)
     
    svm.fit(X_tr, Y_tr)
    Y_pred = svm.predict(X_te)
    print('confusion matrix = \n', confusion_matrix(y_true=Y_te, y_pred=Y_pred))
    print('accuracy = ', accuracy_score(y_true=Y_te, y_pred=Y_pred))
    print('汎化誤差 = ', 1-accuracy_score(y_true=Y_te, y_pred=Y_pred))
    print('precision = ', precision_score(y_true=Y_te, y_pred=Y_pred))
    print('f1 score = ', f1_score(y_true=Y_te, y_pred=Y_pred))
\end{lstlisting}

\begin{lstlisting}[style = py,caption=レポート課題3]
 
\end{lstlisting}

\begin{lstlisting}[style = py,caption=レポート課題4]
 
\end{lstlisting}

%%%%%%%%%%%%%%%%%%%%%%%%%%%%%%%%%%%%%%%%%%%%%%%%%%%%%%%
\end{document}