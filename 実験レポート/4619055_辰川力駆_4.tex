\documentclass[12pt]{jarticle}
\usepackage{TUSIreport}
\usepackage{otf}
%%%%%%%%%%%%%%%%%%
\begin{document}
%%%%%%%%%%%%%%%%%%%%%%%%%%%%%%%%%%%%%%%%%%%%%%%%%%%%%%%%
% 表紙を出力する場合は,\提出者と\共同実験者をいれる
% \提出者{科目名}{課題名}{提出年}{提出月}{提出日}{学籍番号}{氏名}
% \共同実験者{一人目}{二人目}{..}{..}{..}{..}{..}{八人目}
%%%%%%%%%%%%%%%%%%%%%%%%%%%%%%%%%%%%%%%%%%%%%%%%%%%%%%%
\提出者{情報工学実験3}{課題4 画像変換}
{2021}{4}{22}{4619055}{辰川力駆}
%%%%%%%%%%%%%%%%%%%%%%%%%%%%%%%%%%%%%%%%%%%%%%%%%%%%%%%%%
\共同実験者{}{}{}{}{}{}{}{}
%%%%%%%%%%%%%%%%%%%%%%%%%%%%%%%%%%%%%%%%%%%%%%%%%%%%%%%%%
% 表紙を出力する場合はコメントアウトしない
%%%%%%%%%%%%%%%%%%%%%%%%%%%%%%%%%%%%%%%%%%%%%%%%%%%%%%%%%
\表紙出力
%%%%%%%%%%%%%%%%%%%%%%%%%%%%%%%%%%%%%%%%%%%%%%%%%%%%%%%
% 以下はレポート本体,reportmain.tex に書いてある.
% \inputを使っているが,直接書いても良い.
%%%%%%%%%%%%%%%%%%%%%%%%%%%%%%%%%%%%%%%%%%%%%%%%%%%%%%%
\section{実験の要旨}

実験環境を整備した上で、
Jupyther notebookでPython言語とそのライブラリの使い方を理解し、
画像処理の基礎を学習する。

\section{実験の目的}

画像変換の処理を題材に、
画像処理のプログラミングと評価を通じて、
画像処理の基本的な考え方を理解することを目的とする。

第1回目の実験では、濃淡変換の実装と実験を通じて、
Python言語とライブラリの使用方法、
画像処理の基本を理解することを目的とする。

\section{課題1}
\subsection{実験方法}
\subsection{実験結果}
\subsection{検討・考察}

\section{課題2}
\subsection{実験方法}
\subsection{実験結果}
\subsection{検討・考察}

\section{課題3}
\subsection{実験方法}
\subsection{実験結果}
\subsection{検討・考察}


\clearpage

\section{まとめ}



% 参考文献
\begin{thebibliography}{99}

\end{thebibliography}

\clearpage
\appendix
\section{付録}




%%%%%%%%%%%%%%%%%%%%%%%%%%%%%%%%%%%%%%%%%%%%%%%%%%%%%%%
\end{document}